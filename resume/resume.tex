
\documentclass{resume} % Use the custom resume.cls style

\usepackage[left=0.75in,top=0.6in,right=0.75in,bottom=0.6in]{geometry} % Document margins
\usepackage{array}
\usepackage{ragged2e}
\usepackage{hyperref}
\hypersetup{
    colorlinks=true,
    linkcolor=blue,
    filecolor=magenta,
    urlcolor=blue,
}
\newcommand{\tab}[1]{\hspace{.2667\textwidth}\rlap{#1}}
\newcommand{\itab}[1]{\hspace{0em}\rlap{#1}}
\name{Will Langford} % Your name
\address{will.langford@cba.mit.edu}
\address{\href{http://willlangford.me}{willlangford.me}}
\begin{document}

%----------------------------------------------------------------------------------------
%	EDUCATION SECTION
%----------------------------------------------------------------------------------------

\begin{rSection}{Education}

{\bf Massachusetts Institute of Technology}, Cambridge MA \hfill {\em August 2012 - Present}
\\ Advisor: Neil Gershenfeld | Center for Bits and Atoms
\\ PhD Candidate | expected 2019
\\ Master of Science | 2014, Thesis: \href{http://cba.mit.edu/docs/theses/14.08.Langford.pdf}{Electronic Digital Materials}
\smallskip
\\{\bf Tufts University}, Medford MA \hfill {\em August 2008 - May 2012}
\\ Bachelor of Science in Mechanical Engineering \hfill { Overall GPA: 3.81}
\\ Minor in Engineering Management

\begingroup\leftskip=0.7cm

{\bf Honors and Awards}: Vincent Manno Leadership Award, The Prize Scholarship of the Class of 1882, O’Leary Design Award, Mead Jonathan Taylor Prize, Summa Cum Laude, Tau Beta Pi, Dean’s List (all semesters)

\endgroup

\end{rSection}

\begin{rSection}{Research Projects}

{\bf A Discrete Approach to Robotic Construction} \hfill {\em ongoing}
\\My research explores assembly-based fabrication methods that enable the construction of a wide variety of robots from a small set of millimeter-scale parts.

{\bf Digital Material Assembler} \hfill {\em 2016}
\\I developed an automated assembly machine that is able to build electronic components by placing individual conductive and insulating building blocks. \href{http://www.willlangford.me/stapler/index.html}{Project link}

{\bf Desktop Digital Fabrication Tools} \hfill {\em 2014 - Present}
\\I've designed and built a number of custom desktop-scale digital fabrication tools that I use in my research.
\\\href{https://langfordw.pages.cba.mit.edu/desktopWEDM}{Desktop Wire EDM} $\diamond$
\href{http://www.willlangford.me/tabletopInstron/index.html}{Micro-Materials Tester} $\diamond$
\href{http://fab.cba.mit.edu/classes/865.15/people/will.langford/9_processors/index.html}{Punch Press} $\diamond$
\href{http://www.willlangford.me/foldafab/index.html}{Foldafab} $\diamond$
\href{http://www.willlangford.me/handheld_cnc/index.html}{Handheld CNC}

{\bf Electronic Digital Materials} \hfill {\em 2014}
\\My masters work demonstrated and characterized the assembly of complex electronic functionality from a standardized set of conductive, insulating, resistive, and semiconducting building blocks. \href{http://www.willlangford.me/electronic_digital_materials/index.html}{Project link}

{\bf FODHippo: Autonomous airport runway debris removal} \hfill {\em 2012}
\\As my senior design project, my team and I developed and prototyped a robotic debris removal system for airport runways. We were awarded second place in FAA Design Competition for Universities. \href{http://www.willlangford.me/hippo/index.html}{Project link}



\end{rSection}
%--------------------------------------------------------------------------------
%    Projects And Seminars
%-----------------------------------------------------------------------------------------------
\begin{rSection}{Experience}
{\bf Cardibo Inc.}, {\em Hardware Developer} \hfill {\em Summer 2011}
\\Designed a suite of wireless sensor nodes for use in gym equipment monitoring services.
\smallskip
\\{\bf Center for Engineering Education and Outreach}, {\em R\&D Associate} \hfill {\em 2010 - 2011}
\\Designed a circuit board that allows Lego NXT motors and sensors connected and controlled
from an Arduino microcontroller.
\\Developed a method of programming Arduino microcontrollers using Labview. Assisted in
testing software, sourcing parts for a kid-friendly robotics kit, and web development.
\smallskip
\\{\bf Makerbot Industries}, {\em Summer R\&D Associate} \hfill {\em Summers 2009 - 2011}
\\Researched and prototyped “Dual-Extrusion” technologies for Makerbot 3D printers.
\\Designed, tested, programmed, and launched the “Unicorn” pen plotter tool-head.
\\Supported development, production, and distribution of first generation desktop 3D printers.
\smallskip
\\{\bf MIT Non-Newtonian Fluids Lab}, {\em Research Associate} \hfill {\em Jan-May 2011}
\\Implemented control circuitry for a linear vertical stage to be used in viscosity experiments.

\end{rSection}
%----------------------------------------------------------------------------------------
%	TECHNICAL STRENGTHS SECTION
%----------------------------------------------------------------------------------------
\newpage
\begin{rSection}{Technical Skills}
% \begin{tabular}{ @{} >{\bfseries}l @{\hspace{6ex}} l }
% \begin{tabular}{ @{} >{\bfseries}l @{\hspace{4ex}}p{0.7\columnwidth}l }
\begin{tabular}{ >{\bfseries} l p{0.7\columnwidth} l}
Fabrication \ & CNC milling/turning (3/4/5-axis), wire-EDM, waterjet, laser micromachining, 3D printing (FDM, SLA), manual lathe \& mill, HSMWorks, Fusion360 CAM
\smallskip
\\Imaging/Measurement \ & Materials strength testing (Instron), X-ray CT, SEM, confocal microscopy, 3D laser scanning
\smallskip
\\Embedded Systems \ & Atmel AVR (ATtiny, ATmega, Xmega), ARM, Arduino
\smallskip
\\2D/3D Design \ & Fusion360, Solidworks, Inventor, Rhino, Eagle, KiCad, Illustrator
\smallskip
\\Programming \ & Python, Javascript (incl. ThreeJS, Node, Electron), HTML, Labview, Excel VBA, Git
\smallskip
\\Simulation/Modeling \ & COMSOL Multiphysics (electromagnetics, electrostatics, structural, thermal), Numpy/Scipy, MATLAB (incl. Simulink)
\end{tabular}

\end{rSection}

%----------------------------------------------------------------------------------------
%	WORK EXPERIENCE SECTION
%----------------------------------------------------------------------------------------

\begin{rSection}{Teaching \& Leadership}
Deployed \href{http://www.fabfoundation.org/}{Fablabs} in Saudi Arabia, \href{https://foundation.ayb.am/en/bepart/first-two-fab-labs-in-armenia}{Armenia}, \href{https://www.engineersrule.com/fablab-network-spreads-rwanda-solidworks-support/}{Rwanda}, and \href{http://www.fablab.bt/}{Bhutan}, 2014-2018

TA for \href{http://fab.cba.mit.edu/classes/863.18/}{How to Make (Almost) Anything}, a graduate course at MIT, 2013-2018

TA for \href{http://fab.cba.mit.edu/classes/865.18/index.html}{How to Make Something That Makes (Almost) Anything}, a graduate course at MIT, 2018

TA for \href{http://fab.cba.mit.edu/classes/864.17/index.html}{The Nature of Mathematical Modeling}, a graduate course at MIT, 2017

TA for \href{http://fab.cba.mit.edu/classes/862.16/index.html}{The Physics of Information Technology}, a graduate course at MIT, 2016

TA for Electronic Musical Instrument Design, an undergraduate course at Tufts University, 2011

TA for Intro to Robotics and Mechatronics, an undergraduate course at Tufts University, 2011

{\textcolor{black}{Founder and president of Tufts Robotics Club}, 2008-2012}

\begingroup\leftskip=0.7cm
Director of the Tufts Botlab, a student run robotics and fabrication lab

Lead teams which placed 1st on the Trinity Firefighting Robotics Olympiad Exam twice

Granted Student Life Imagination Award for developing and conducting student robotics workshops including “CNC Cupcake Frosting,” “Sumobot Competition,” “Friendly Robotics,” and “Toy-hacking Elmo”

Mentored Medford High School and Melrose High School robotics teams

Lead teams which developed an automated hydroponics gardening systems, 12lb Battlebots for Robot Conflict competitions, and a Mars rover robot

\endgroup

\end{rSection}

%----------------------------------------------------------------------------------------
% Publications
%----------------------------------------------------------------------------------------
\begin{rSection}{Publications}
Langford W, Gershenfeld N. \href{https://docs.lib.purdue.edu/iutam/presentations/abstracts/45}{Discretely Assembled Compliant Mechanisms}, Proceedings of the IUTAM Symposium Architectured Materials Mechanics, 2018

Langford W, Ghassaei A, Jenett B, Gershenfeld N. \href{http://cba.mit.edu/docs/papers/17.04.11.SelfAssemSpacecraft.pdf}{Hierarchical Assembly of a Self-Replicating Spacecraft}, IEEE Aerospace, 2017

Langford W, Ghassaei A, Gershenfeld N. \href{http://cba.mit.edu/docs/papers/16.07.msec.stapler.pdf}{Automated Assembly of Electronic Digital Materials}, Proceedings of the Manufacturing Science and Engineering Conference, 2016
\sectionskip
\\\textcolor{black}{Book Features}

\begingroup\leftskip=0.7cm
Designing Reality by Neil Gershenfeld, 2018

Active Matter by Skylar Tibbits, 2017

\endgroup

\textcolor{black}{Patents}

\begingroup\leftskip=0.7cm

% Discrete Motion System, 2013

Discrete Assemblers Utilizing Conventional Motion Systems, \href{https://patents.google.com/patent/US10155313B2/}{US10155313B2}, {\em issued 2018}

Self-assembling assemblers %and manipulators
built from a set of primitive blocks, \href{https://patents.google.com/patent/US10155314B2}{US10155314B2}, {\em issued 2018}

Electromagnetic Digital Materials, \href{https://patents.google.com/patent/US20140145522A1}{US20140145522A1}, {\em pending 2011}

% Wireless sensor network for determining cardiovascular machine usage, US20140300211A1, 2011

\endgroup

\end{rSection}

% \newpage

\begin{rSection}{Speaking Events}

\href{http://cba.mit.edu/events/18.06.SCF/}{Symposium on Computational Fabrication}, {\em Assembly Fabrication}, June 2018

\href{https://mitmuseum.mit.edu/program/dimensions-doctor-who}{Dimensions of Doctor Who}, {\em Using Shape-Shifting Matter To Build the TARDIS}, April 2016

\href{http://fab11.fabevent.org/#}{FAB11 Symposium}, {\em Digital Material Assembly}, August 2015

\href{http://cba.mit.edu/events/13.03.scifab/index.html}{The Science of Digital Fabrication}, {\em Micro-Assembly}, May 2013

\end{rSection}

\begin{rSection}{Media}

\href{https://www.youtube.com/watch?v=RaHMDNf56W4}{\em Adam Savage's Maker Tour}, Tested.com 2017

\href{https://vimeo.com/181328010}{\em Fablabs}, WONDROS 2016

\href{https://www.cnn.com/videos/bestoftv/2013/07/17/exp-gps-gershenfeld-3d-printing.cnn}{\em On GPS: Future of Digital Fabrication}, CNN 2013

\href{https://www.americaninno.com/boston/looking-to-frost-cupcakes-deal-with-the-economic-stimulus-plan-tufts-has-some-robots-for-that/}{\em Looking To Frost Cupcakes \& Deal With the Economic Stimulus Plan? Tufts Has Some Robots For That}, BostInno 2012

\href{https://web.archive.org/web/20120718010336/http://enews.tufts.edu/stories/1642/2010/03/31/WillLangford}{\em If you build it...}, Tufts E-News 2010

% Thingiverse Thing of the Week

Make Magazine:
\href{https://makezine.com/2011/09/20/more-about-dualstrusion-from-makerbot/}{Dualstrusion} $\diamond$
\href{https://makezine.com/2010/04/16/hydroponic_herb_garden/}{Bloombot} $\diamond$
\href{https://makezine.com/2010/01/14/3d-printed-bracelet-of-the-future/}{Bracelets} $\diamond$
\href{https://makezine.com/2009/11/30/makerbot-sumobot/}{Sumobots} $\diamond$
\href{https://makezine.com/2009/07/14/print-your-own-glasses/}{Glasses} $\diamond$
\href{https://makezine.com/2009/09/21/printing-braille/}{Braille}

\href{https://www.wired.com/2009/08/makerbot/}{\em 3-D Printers Make Manufacturing Accessible}, Wired 2009

\end{rSection}
\end{document}
